\documentclass[a4paper]{article}
    \usepackage[UTF8]{ctex}
    \usepackage{minted}
    \usepackage{xcolor}
    \usepackage[colorlinks,linkcolor=black]{hyperref}
    \usepackage[margin=1in]{geometry}
    \usepackage{caption}
    \usepackage{graphicx, subfig}
    \usepackage{float}
    \usepackage{fontspec}
    \setmainfont{Times New Roman}
    \setmonofont{Consolas}
    \definecolor{bg}{rgb}{0.9,0.9,0.9}
    \usemintedstyle{manni}
    \setminted{
    linenos,
    autogobble,
    breaklines,
    breakautoindent,
    bgcolor=bg,
    numberblanklines=false,
    }

\begin{document}
    \tableofcontents
    \newpage
    \section{实验介绍}
        \subsection{实验内容}
本次报告涵盖了第二次和第三次实验,其内容分别如下:
            \subsubsection{实验二}
1.使用自己的账号模拟登陆BBS后,修改个人说明档(修改bbs\_set\_sample.py).

2.修改crawler\_sample.py中的union\_bfs函数,完成BFS搜索.

3.修改crawler\_sample.py中的crawl函数,返回图的结构.

4.进一步修改函数,完成网页爬虫(将练习2,3中修改的部分加入crawler.py,注意做异常处理).
            \subsubsection{实验三}
1.实现BloomFilter,并设计一个实验统计你的BloomFilter的错误率(false positive rate).

2.实现一个并行的爬虫,将实验二中的crawler.py改为并行化实现.


        \subsection{实验环境}
操作系统: Ubuntu 18.04 LTS

Python版本: 3.7
        \subsection{所需技能}
实验二主要涉及运用Python模拟网络请求,这需要具备一定的网络知识.而其后对网页内容进行解析,并根据不同策略进行爬取,则需要
基本的算法知识.

实验三涉及到Hash Function, Hash Table, BitMap以及Bloom Filter的原理以及实现.在编写多线程并行程序时,
还需要对操作系统,阻塞以及锁等概念有些了解.
    \newpage
    \section{实验过程}
        \subsection{Basic Crawler}
            \subsubsection{BBS修改个人说明档}
该练习主要在于了解用POST请求提交数据的流程,较为简单.

实验提供的指导是用urllib库来模拟网络请求,cookielib库来操作cookie,操作较为繁琐.事实上,由于本练习仅需要保持登录状态,
并不需要涉及直接操作cookie的细节,因此可以使用requests库提供的Session()对象.如下:
\begin{minted}{python}
def bbs_set(id, pw, text):

    s = requests.Session()
    login_data = {
        "id": id,
        "pw": pw,
        "submit": "login",
    }
    login_url = "https://bbs.sjtu.edu.cn/bbslogin"
    login_request = s.post(login_url, login_data)
    assert login_request.status_code == 200, "Failed to log in."
    update_data = {
        "text": text,
        "type": "update",
    }
    update_url = "https://bbs.sjtu.edu.cn/bbsplan"
    update_request = s.post(update_url, update_data)
    content = s.get('https://bbs.sjtu.edu.cn/bbsplan').content
    soup = BeautifulSoup(content, features="html.parser")
    print(str(soup.find('textarea').string).strip())
\end{minted}

值得注意的是对提交本文的编码,需要进行GBK编码.

            \subsubsection{BFS}
由于此处未考虑运行的效率,BFS函数的实现相当简单:
\begin{minted}{python}
def union_bfs(a, b):
    for e in b:
        if e not in a:
            a.insert(0, e)
\end{minted}
            \subsubsection{爬取与解析页面}
在爬取页面时,为了避免程序中途因为网络原因而崩溃,需要捕获异常以增强健壮性:
\begin{minted}{python}
def get_page(page):
    try:
        r = requests.get(page, timeout=10)
    except:
        return None
    content = r.content
    return content
\end{minted}

而在解析页面时,可以复用实验一时所写的\mintinline{python}{BaseParser}类,而无需重复编码:
\begin{minted}{python}
from parsers import BaseParser
def get_all_links(content, page):
    parser = BaseParser()
    url_set = parser.parse_url(content, page)
    links = [u for u in url_set]
    return links
\end{minted}
            \subsubsection{网页编码自动检测}
在实验一的报告中,我将网页编码的问题暂时搁置.而在这次实验中,由于要爬取的目标不再是单一的网站,而是互联网上各式各样的
网站,其编码方式各异,网页编码的问题便无法回避了.

\mintinline{python}{add_page_to_folder}函数将爬取的网页保存到本地文件(UTF-8编码),其间涉及到两次编码解码:一是对网页内容解码,
二是在写入文件时编码.第二步并无大碍,但第一步,如何对不同编码的网页解码呢?

事实上,Python提供了chardet库专门用以检测文本的编码,只需在解码前先检测出可能的编码,便可以从容应对大多数的情况.至于少数
检测不出编码方式的网页文本,只能按照默认的UTF-8来处理,并忽略解码时的错误.如下:

\begin{minted}{python}
import chardet

def add_page_to_folder(page, content): 
    #some code omitted
    #...
    f = open(os.path.join(folder, filename), 'w', encoding="utf8")
    encode = chardet.detect(content)
    if not encode["encoding"]:
        print("Failed to detect encoding.({url},{encode})".format(url=page, encode=encode))
    f.write(content.decode(encode["encoding"] or "utf8", errors="ignore"))
    f.close()
\end{minted}
        \subsection{Bloom Filter}
            \subsubsection{Hash Function}
网络上有许多种公开可用的哈希函数,像本次实验提供的General Purpose Hash Function Algorithms Library中就已有10余种哈希函数,
除此之外,更是有诸如MD5,SHA1等常用于加密的哈希函数.

对于Bloom Filter,哈希函数的选用十分关键,关涉到运行的效率与检测的正确性.考虑到我们的过滤器仅仅用于爬虫的网址去重,相较于安全性,
更注重运行的速度.

那么,如何找到一个高效率的哈希函数呢?诚然,可以从算法角度进行分析,但既然我们手中已有现成的代码实现,设计一个简单进行测试是最直接的做法.
为此,我编写了HashTest.py,用以比较不同的哈希函数的运行效率:
\begin{minted}{python}
from GeneralHashFunctions import *
import mmh3
import random
import time
from string import ascii_letters as letters

def test_hash_efficiency(hash_function, test_strings):
    start_time = time.time()
    for s in test_strings:
        hash_function(s)
    finish_time = time.time()
    return finish_time - start_time

test_strings = ["".join(random.sample(letters, 32)) for i in range(100000)]
hash_functions = [hash, mmh3.hash, JSHash, PJWHash, ELFHash, BKDRHash, SDBMHash, DJBHash, DEKHash, BPHash, FNVHash, APHash]
result = []

for f in hash_functions:
    result.append((f.__name__, test_hash_efficiency(f, test_strings)))
    print("{0:<9}: {1:.8f}s".format(*result[-1]))
print("*********Result Sorted By Time**********")
result.sort(key=lambda x: x[1])
for r in result:
    print("{0:<9}: {1:.8f}s({2:.3f}x)".format(*r, r[1]/result[0][1]))
\end{minted}

随机生成了100000个长度为32的字符串,然后对其用不同的哈希函数进行散列处理,比较运行时间,最终得到的结果如下:
\begin{minted}{python}
*********Result Sorted By Time**********
hash     : 0.00897479s(1.000x)  # Built-in Hash
hash     : 0.02493358s(2.778x)  # mmh3.hash
BPHash   : 0.73630977s(82.042x)
BKDRHash : 0.83501196s(93.040x)
DJBHash  : 0.88225102s(98.303x)
FNVHash  : 0.99035144s(110.348x)
DEKHash  : 1.01107359s(112.657x)
SDBMHash : 1.23072314s(137.131x)
PJWHash  : 1.63085604s(181.715x)
JSHash   : 2.06598854s(230.199x)
APHash   : 2.06879592s(230.512x)
ELFHash  : 2.11861658s(236.063x)
\end{minted}

可见,从运行效率上来说,不同哈希函数的差距巨大(相差最大236倍),其中,Python自带的\mintinline{python}{hash()}最优,
murmurhash3紧随其后.但是,考虑到在Bloom Filter中需要对同一个哈希函数加以不同的种子(当然也可以用不同的哈希函数,
但那样较为繁琐),而Python自带的\mintinline{python}{hash()}并不支持这么做,故最终我选择了murmurhash3,近年兴起
的一种高效率哈希函数.
            \subsubsection{Bit Array}
在实验提供的Python中,已实现了\mintinline{python}{Bitarray}类,但由于其编写于Python 2.x环境下,并且存在一个严重的Bug,
需要对其进行一些修正.修改后的\mintinline{python}{Bitarray}类如下:
\begin{minted}{python}
class Bitarray:
    def __init__(self, size):
        """ Create a bit array of a specific size """
        self.size = size
        self.bitarray = bytearray(size//8 + 1) # Attention Here!!!

    def set(self, n):
        """ Sets the nth element of the bitarray """
        index = n // 8
        position = n % 8
        self.bitarray[index] = self.bitarray[index] | 1 << (7 - position)

    def get(self, n):
        """ Gets the nth element of the bitarray """
        index = n // 8
        position = n % 8
        return (self.bitarray[index] & (1 << (7 - position))) > 0
\end{minted}

需要注意的是,原程序对\mintinline{python}{bytearray}的大小设置有误,运行时有概率导致下标越界.
            \subsubsection{BloomFilter类实现}
对于哈希函数个数k、位数组大小m、加入的字符串数量n,根据一定的数学推导,可知当 k = ln(2) * m/n 时出错的概率是最小的.
而设定 m = 20 * n,则出错的概率为0.0000889,对于爬虫而言完全可以接受.

因此,在类初始化时,既允许同时指定n,m,k,也可以允许只指定n,其余值按照上述规则计算.

而在对哈希函数列表初始化时,用到了Python具有的函数式语言特点,通过自行实现的类的私有方法
\mintinline{python}{__hash_function_wrapper}来对哈希函数进行
处理,方便之后的调用:

\begin{minted}{python}
import mmh3     # third-party library for murmurhash3
from functools import partial
import math

class BloomFilter(object):

    def __hash_function_wrapper(self, hash_func, seed, mod):
        def f(s):
            return partial(hash_func, seed=seed)(s) % mod
        return f

    def __init__(self, potential_num, bitarray_size=None, hash_num=None):
        self.potential_num = potential_num
        self.bitarray_size = bitarray_size or 20 * potential_num
        self.hash_num = hash_num or int(math.log(2)*self.bitarray_size/self.potential_num)
        self.hash_functions = [self.__hash_function_wrapper(
            mmh3.hash, seed=i, mod=self.bitarray_size
            ) for i in range(self.hash_num)]
        self.bitarray = Bitarray(self.bitarray_size)
        self.count = 0
\end{minted}

在初始化完成之后,Bloom Filter所需实现的操作为加入元素\mintinline{python}{add}
与检查元素是否已加入\mintinline{python}{check},至于\mintinline{python}{__len__}与
\mintinline{python}{parameters},是为了方便外界获知Bloom Filter的状态.
\begin{minted}{python}
    def add(self, string):
        for hash_func in self.hash_functions:
            self.bitarray.set(hash_func(string))
        self.count += 1

    def check(self, string):
        return all(self.bitarray.get(hash_func(string))
                   for hash_func in self.hash_functions)


    def __len__(self):
        return self.count

    def parameters(self):
        return self.potential_num, self.bitarray_size, self.hash_num
\end{minted}
        \subsection{Multi-threading}
在原有的crawler.py程序基础上,为了使其能够以多线程的方式并发爬取,需要修改\mintinline{python}{crawl}函数.

为了不引入过多的全局变量污染程序,利用Python的函数式特性:闭包,我在\mintinline{python}{crawl}函数内部定义了
\mintinline{python}{crawl_single_page}函数作为多线程的worker.大致如下:
\begin{minted}{python}
def crawl(seed, max_page, thread_num=4):

    def crawl_single_page():
        pass
    url_queue = Queue()
    lock = Lock()
    crawled = BloomFilter(max_page)
    graph = {}
    global count
    count = 0
    url_queue.put(seed)
    for i in range(thread_num):
        t = Thread(target=crawl_single_page)
        t.setDaemon(True)
        t.start()
    url_queue.join()
    return graph, crawled
\end{minted}

需要注意的是,不同于实验二,这里的\mintinline{python}{crawled}不是列表,而是我们所实现的Bloom Filter.

至于\mintinline{python}{crawl_single_page}的具体实现,与实验二基本无异,不同之处在于锁的引入,避免多线程
导致变量读写的异常.
\begin{minted}{python}
def crawl_single_page():
    global count
    while count < max_page:
        url = url_queue.get()
        if not crawled.check(url):
            print("#{0:<4} {1}".format(count+1, url))
            content = get_page(url)
            outlinks = get_all_links(content, url)
            with lock:
                if count >= max_page: #attention here
                    print("Drop!")
                    break
                graph[url] = outlinks
                for l in outlinks:
                    url_queue.put(l)
                crawled.add(url)
                count += 1
            add_page_to_folder(url, content)
        url_queue.task_done()
\end{minted}

留意以上代码,为了确保最终爬取的页面数量与设定一致,在爬取完页面后,需要丢弃由于多线程导致的冗余页面(如目标是100个页面,但在爬取
完99个页面后,4个线程从队列中取出一个页面进行爬取).

然而,由于我们在爬取到足够数量页面之后就停止爬取,队列中往往会存留尚未爬取的页面,这就导致了在线程结束后依旧阻塞,程序无法终止.因此,
在主程序中,应对每个Thread进行join操作,而非对Queue.
\begin{minted}{python}
threads = []
for i in range(thread_num):
    t = Thread(target=crawl_single_page)
    threads.append(t)
    t.setDaemon(True)
    t.start()
for t in threads:
    t.join()
\end{minted}
    \newpage
    \section{总结与分析}
经过实验二与实验三,相较于实验一中简单的爬虫,已有不少提升,如Bloom Filter和多线程对运行效率的显著提升,以及许多诸如
网页编码自动检测等细节的优化.在进行实验的过程中,课程提供的指引起到了很大的作用,但在某些情形下,还是需要自己查找资料并
动手实践来解决问题,这样让我学到了不少新的内容.

此外,每次实现一个功能,总是有不止一种可供选择的方案.有的方案简单而粗暴,却缺乏美感(譬如引入一大堆全局变量),我总是试图
避免出现这样的情况,因而或许写出了一些不太常规的代码.

虽然多费了时间,但挺有趣的.
\end{document}
