\documentclass[a4paper]{article}
    \usepackage[UTF8]{ctex}
    \usepackage{minted}
    \usepackage{xcolor}
    \usepackage[colorlinks,linkcolor=black]{hyperref}
    \usepackage[margin=1in]{geometry}
    \usepackage{caption}
    \usepackage{graphicx, subfig}
    \usepackage{float}
    \usepackage{fontspec}
    \usepackage{booktabs}
    \setmainfont{Times New Roman}
    \setmonofont{Consolas}
    \definecolor{bg}{rgb}{0.9,0.9,0.9}
    \usemintedstyle{manni}
    \setminted{
    linenos,
    autogobble,
    breaklines,
    breakautoindent,
    bgcolor=bg,
    numberblanklines=false,
    }

\begin{document}
    \tableofcontents
    \newpage
    \section{实验介绍}
        \subsection{实验内容}
1.将img1.png和img2.png两幅图像以彩色图像方式读入,并计算颜色直方图.

2.将img1.png和img2.png两幅图像以灰度图像方式读入,并计算灰度直方图和梯度直方图.
        \subsection{实验环境}
操作系统:Ubuntu 18.04 LTS

Python版本:3.7

NumPy:1.14.5

OpenCV-Python:3.4.3.18

Matplotlib:2.2.2
    \newpage
    \section{实验过程}
        \subsection{灰度直方图}
            \subsubsection{原理}
灰度图像$I(x,y)$的灰度直方图定义为各灰度值像素数目的相对比例,反映了图像的明暗程度.
图像中灰度值为i的像素总个数为:
$$N(i) = \sum_{x=0}^{W-1}\sum_{y=0}^{H-1}(I(x,y)==i?1:0)$$
则灰度直方图:
$$H(i)=\frac{N(i)}{\sum_{j=0}^{255}N(j)},i=0,\dots,255$$
            \subsubsection{程序实现}
在Python中,可以使用NumPy进行向量化编程,避免对图像进行二层循环.例如,\mintinline{python}{np.bincount()}函数
可以方便地实现计数操作.在得到直方图数据后,使用Matplotlib绘图即可.
\begin{minted}{python}
def grayscale_hist(image_path):
    img = cv2.imread(image_path, cv2.IMREAD_GRAYSCALE)
    hist = np.bincount(img.ravel(), minlength=256)
    hist = hist / np.sum(hist)
    plt.bar(range(0, 256), hist, width=1.0)
    plt.title("Grayscale Histogram of {}".format(image_path))
    plt.show()
\end{minted}
        \subsection{颜色直方图}
            \subsubsection{原理}
彩色图像$I(x,y,c)$的颜色直方图是它三个颜色分量总能量的相对比例,反映了图像的总体色调分布.
某一颜色分量的总能量:
$$E(c)=\sum_{x=0}^{W-1}\sum_{y=0}^{H-1}I(x,y,c)$$
某一颜色分量的能量相对比例:
$$H(c)=\frac{E(c)}{\sum_{i=0}^{2}E(i)}$$

            \subsubsection{程序实现}
同样的,利用NumPy来操作图像对应的张量,并使用Matplotlib来绘制直方图.为了直观地表现色彩分布,RGB的三个
分量在直方图中分别用对应的颜色表示.
\begin{minted}{python}
def color_hist(image_path):
    img = cv2.imread(image_path, cv2.IMREAD_COLOR)
    hist = np.sum(img.reshape(-1, 3), axis=0)
    hist = hist / np.sum(hist)
    plt.bar(["Blue", "Green", "Red"], hist, color=["blue", "green", "red"])
    plt.title("Color Histogram of {}".format(image_path))
    plt.show()
\end{minted}
        \subsection{梯度直方图}
            \subsubsection{原理}
灰度图像$I(x,y)$的梯度直方图定义为其各像素点的梯度值(边界像素点除外),反映了图像纹理的复杂程度.
一个像素点的梯度定义为:
$$I_{x}(x,y)=\frac{\partial I(x,y)}{\partial x}=I(x+1,y)-I(x-1,y)$$
$$I_{y}(x,y)=\frac{\partial I(x,y)}{\partial y}=I(x,y+1)-I(x,y-1)$$
而梯度强度定义为:
$$M(x,y)=\sqrt{I_{x}(x,y)^2+I_{y}(x,y)^2}$$
由于$-255\le I_{x},I_{y} \le 255$,$0 \le M(x,y) \le 255\sqrt{2} < 361$,则将梯度强度均匀分成361个
区间,定义$(x,y)$处的像素所在区间为:
$$B(x,y) = i, i\le M(x,y) < i+1, 0\le i\le 360$$
故落在第i个区间总的像素数目为:
$$N(i) = \sum_{x=1}^{W-2}\sum_{y=1}^{H-2}(B(x,y)==i?1:0)$$
比例为:
$$H(i)=\frac{N(i)}{\sum_{j=0}^{360}N(j)}$$
            \subsubsection{程序实现}
在使用NumPy操作图像张量时,需要注意其中的数据类型:通过OpenCV读取的图像中默认\mintinline{python}{dtype="uint8"},
即为8位无符号整型数,在进行求梯度操作时会出现溢出的问题,导致最终出现错误的结果.

因此,在读取图像之后,需要改变其数据类型为\mintinline{python}{"int32"}.同时,在计算出\mintinline{python}{M}后,也应当
将其从浮点类型转换为整型数,以方便计数.
\begin{minted}{python}
def gradient_hist(image_path):
    img = cv2.imread(image_path, cv2.IMREAD_GRAYSCALE)
    img = img.astype("int32")
    Ix = (img[:, 2:] - img[:, :-2])[1:-1, :]
    Iy = (img[2:, :] - img[:-2, :])[:, 1:-1]
    M = np.sqrt(Ix * Ix + Iy * Iy).astype("int32")
    hist = np.bincount(M.ravel(), minlength=361)
    hist = hist / np.sum(hist)
    plt.bar(range(0, 361), hist, width=1.0)
    plt.title("Gradient Histogram of {}".format(image_path))
    plt.show()
\end{minted}
    \subsection{主程序}
在完成以上绘制3种直方图的函数后,只需在主程序中分别调用即可:
\begin{minted}{python}
import cv2
import numpy as np
import matplotlib.pyplot as plt


if __name__ == '__main__':
    images_list = ["img1.png", "img2.png"]
    histograms_list = [color_hist, grayscale_hist, gradient_hist]
    for image in images_list:
        for histogram in histograms_list:
            histogram(image)
\end{minted}

绘制的直方图见随报告一起提交的图片文件.
    \newpage
    \section{总结与分析}
本次实验中学习了基本的图像特征提取,同时练习了NumPy的使用.
\end{document}
