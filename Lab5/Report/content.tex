\documentclass[a4paper]{article}
    \usepackage[UTF8]{ctex}
    \usepackage{minted}
    \usepackage{xcolor}
    \usepackage[colorlinks,linkcolor=black]{hyperref}
    \usepackage[margin=1in]{geometry}
    \usepackage{caption}
    \usepackage{graphicx, subfig}
    \usepackage{float}
    \usepackage{fontspec}
    \setmainfont{Times New Roman}
    \setmonofont{Consolas}
    \definecolor{bg}{rgb}{0.9,0.9,0.9}
    \usemintedstyle{manni}
    \setminted{
    linenos,
    autogobble,
    breaklines,
    breakautoindent,
    bgcolor=bg,
    numberblanklines=false,
    }

\begin{document}
    \tableofcontents
    \newpage
    \section{实验介绍}
        \subsection{实验内容}
        本次报告包括了第四次和第五次上机实验的内容,如下:
            \subsubsection{实验四}
实现一个中文网页索引与搜索程序,爬取一定数量(>5k)的中文网页(可利用之前实验爬取的网页),修改IndexFiles.py和SearchFiles.py,
对这些中文网页建立索引并进行搜索,搜索时需要打印出检出文档的路径、网页标题、url等.

即doc的Field中需要有name(文件名),path(文件路径),title(网页标题),url(网页地址),contents(索引的文件内容).

            \subsubsection{实验五}
1.模拟实现搜索引擎的"site:"功能(对搜索的网站进行限制),提示:可以在原先的索引上添加一个可以索引的domain域,来对网址所在的域名进行索引.

2.实现一个图片索引:新建一个索引,输入文本,输出相关的图片地址,图片所在网页的网址,图片所在网页的标题等.
(做图片索引时最好选定某个你感兴趣的网站爬取,比如只对糗事百科上的图片进行索引,这样可以对特定网站的结构进行分析,让搜索结果更精确.)
        \subsection{实验环境}
操作系统: Ubuntu 18.04 LTS

Python版本: 3.7

PyLucene版本: 7.4.0
        \subsection{所需技能}
实验四主要涉及PyLucene的安装以及基本使用方法,需要对Linux环境下软件的编译与安装有所了解,同时也需要理解信息检索的一些基本概念,
譬如中文分词与倒排索引.

实验五对PyLucene进一步深入学习,实现多条件检索以及图片检索,需要对PyLucene的工作机制有更多的认识.
    \newpage
    \section{实验过程}
        \subsection{准备实验环境}
实验指引提供了两种安装PyLucene的方法: 通过conda安装PyLucene 4.10.0或者手工编译安装PyLucene. 考虑到经过多年的迭代,PyLucene最新版本已然是
7.4.0,和conda源相距了3个主版本,不免觉得后者陈旧. 于是,我决定手工编译安装最新版本的PyLucene.

然而,关于编译安装PyLucene的方法,实验指引所提供的方法并不十分有效(按照指引多次尝试均告以失败).参照PyLucene官方提供的安装方法
\footnote{https://lucene.apache.org/pylucene/install.html}
,以及
摸索尝试,最终我找到了完全可行且同时适用于Python 2和Python 3的方法,如下:
            \subsubsection{安装JDK和Ant}
实验指引给出的命令是:\mintinline{bash}{sudo apt-get install default-jdk},但经过实验,由于编译JCC的主文件setup.py的缺陷,
使用open-jdk会存在编译失败的可能.在粗略查看了setup.py的源代码后,我发现其默认使用的JDK是Oracle JDK 8(这一点似乎甚至写死在代码里了),
那么,与其费力修改,不如按照要求,安装Oracle JDK 8. 至于如何安装,大致过程是从官网下载后手动安装,在此不表,需要注意的是,
安装后需要将JDK路径等加入环境变量,同时设置系统的默认JDK
\footnote{设置方法可参照 https://www.digitalocean.com/community/tutorials/how-to-install-java-with-apt-on-ubuntu-18-04\#setting-the-java\_home-environment-variable}
(如果安装了多个JDK).

至于安装Ant,则较为简单,只需\mintinline{bash}{sudo apt install ant}.

            \subsubsection{编译安装JCC}
实验指引给出的命令是:\mintinline{bash}{sudo easy_install jcc},但这样的方法并不奏效,并且无法为系统中不同的Python安装JCC.因此,
更好的办法是手动编译安装.

在从PyLucene官网下载了PyLucene源文件后,解压,进入\mintinline{bash}{./jcc/}文件夹内,根据需要安装JCC的Python,尝试运行:
(其中的python3可换成python2,抑或系统内其他的Python)
\begin{minted}{bash}
    python3 setup.py build
    sudo python3 setup.py install
\end{minted}

但这么做存在失败的可能,setup.py可能无法正确的检测到系统内的JDK,因此,如果失败,需编辑setup.py文件,
在约70行附近,根据实际的JDK路径,修改如下内容后再次尝试:
\begin{minted}{python}
JDK = {
    'darwin': JAVAHOME or JAVAFRAMEWORKS,
    'ipod': '/usr/include/gcc',
    'linux': '/usr/lib/jvm/java-8-oracle', #edit this line!
    'sunos5': '/usr/jdk/instances/jdk1.6.0',
    'win32': JAVAHOME,
    'mingw32': JAVAHOME,
    'freebsd7': '/usr/local/diablo-jdk1.6.0'
}
\end{minted}
        \subsubsection{编译安装PyLucene}
回到PyLucene的主文件夹内,修改Makefile文件,无需理会其中的系统信息,直接在文件开始添加如下几行:
\begin{minted}{makefile}
ANT=ant
PYTHON=python3
JCC=$(PYTHON) -m jcc
NUM_FILES=8
\end{minted}

当然,这么做的前提是在正确的设置了环境变量.而后编译安装即可:
\begin{minted}{bash}
make
make install
\end{minted}
        \subsection{爬取数据}
            \subsubsection{准备工作}
在前几次的实验中,我们分别实现了基本的crawler以及parser,并为其实现了诸如多线程,BloomFilter,自动检测编码等功能.
为了方便后续的使用,我对原有的代码进行了重构与整合,作为工具类,封装为一个Python Package.其结构如下:
\begin{minted}{python}
my_utils
    __init__.py
    bloom_filter.py
        class Bitarray()
        class BloomFilter()
    parsers.py
        class BaseParser()
        class QSBKParser()
    crawlers.py
        class BaseCrawler()
        class MultiThreadingCrawler()
\end{minted}

其中改动较大的部分为\mintinline{python}{MultiThreadingCrawler()}:增加了Session池,哈希文件名等功能.

Session池的目的在于提高爬取的效率,无需为每次请求生成单独的Session,同时也对被爬取网站更加友好,避免同时产生多个连接,
从而降低其负担.采用的实现方式较为简易:初始化时生成一定数量的Session并放入池中,每次请求时随机从池中取出一个Session使用,
代码如下:
\begin{minted}{python}
def __init__(self, thread_num=4, headers=None, session_num=10,
             index_file=None, data_folder=None, debug=False, verbose=True):
    # ...
    self.sessions = list()
    for i in range(session_num):
        self.sessions.append(requests.Session())
    # ...

def get_html(self, url, headers=None):
    assert len(self.sessions) != 0, "There's no Session available!"
    current_session = random.choice(self.sessions)
    raw_html = current_session.get(url, headers=headers).content
    # ...
\end{minted}

哈希文件名的目的在于规范保存文件时的文件名,避免出现特别长的文件名(爬取过程中会遇到某些非常长的URL).特别长的文件名可能会遇到
系统层面的不支持,并且在后续的实验中,PyLucene创建索引时遇到长文件名会报错.因此,可以将URL经过哈希处理后再作为文件名,
则可保证文件名的规范性:
\begin{minted}{python}
    def __hash_filename(self, s):
        md5 = hashlib.md5()
        md5.update(s.encode("utf8"))
        return md5.hexdigest()+".html"
\end{minted}
        \subsubsection{爬取HTML数据}
根据实验要求,需要爬取至少5000个网页并存储在本地,为了提高爬取的效率,使用多线程爬取器,即\mintinline{python}{MultiThreadingCrawler()}进行
较为合适.然而出于爬虫的礼貌性,过于高频地访问同一主机又是不合适的.因此,需要为每次请求设定一定的的时间间隔.

此外,为了爬取到类别丰富的网页,同时也为了避免给同一网站造成过大的访问压力,可以设定爬取起点为导航类网站,如hao123.从一定意义上思考,
这相当于对整个互联网进行了取样,保证了爬取图的连通性.

结合以上几点,爬取HTML数据的脚本(crawl\_data.py)如下:
\begin{minted}{python}
from my_utils.crawlers import MultiThreadingCrawler
import sys
import os

if __name__ == '__main__':
    seed = "https://www.hao123.com/"
    max_page = 5000
    thread_num = 16
    sleeping = 5
    session_num = 32
    data_dir = "data"
    if not os.path.exists(data_dir):
        os.mkdir(data_dir)
    index_file = os.path.join(data_dir, "index.txt")
    data_folder = os.path.join(data_dir, "html_data")
    headers = {
        # omitted
    }
    my_crawler = MultiThreadingCrawler(thread_num, headers, session_num,
                                       index_file, data_folder, debug=False, verbose=False)
    my_crawler.crawl_from(seed, max_page, thread_num, sleeping)
\end{minted}

几十分钟之后,爬取完成.
        \subsubsection{爬取图片数据}
爬取图片本身并不困难,只需提取出网页中的图片元素即可,然而出于检索的目的,需要对图片的内容有所"了解",这就需要从
图片所在网页提取相关的描述性信息.考虑到不同网页结构的差异,在本次实验中,我仅仅针对单一网站进行图片检索.利用之前实验
中实现的糗事百科解析器,可以比较方便地完成爬取(crawl\_pic.py):
\begin{minted}{python}
import os
from my_utils.crawlers import BaseCrawler
from my_utils.parsers import QSBKParser

if __name__ == '__main__':
    next_page = "https://www.qiushibaike.com/pic/"
    data_dir = "pic_data"
    if not os.path.exists(data_dir):
        os.mkdir(data_dir)
    index_file = os.path.join(data_dir, "index.txt")
    data_folder = os.path.join(data_dir, "html_data")
    headers = {
        # omitted
    }
    crawler = BaseCrawler(headers=headers)
    parser = QSBKParser()
    while next_page:
        print(next_page)
        html_content = crawler.get_html(next_page)
        docs, next_page = parser.parse_page(html_content, current_url=next_page)
        with open(index_file, mode="a", encoding="utf8") as index:
            for post in docs.values():
                index.write("{img}\t{content}\n".format(img=post["img_url"],
                    content=post["content"]))
    print("Done!")
\end{minted}
        \subsection{建立索引}
            \subsubsection{提取HTML文档内容}
实验指引中给出了两种方法:使用BeautifulSoup或nltk库.但经过实验,后者无法正常使用,而前者的效果并不理想.以百度首页为例,其提取
的文本为:
\begin{minted}{text}
html \n STATUS OK   百度一下,你就知道                                           
新闻   hao123   地图   视频   贴吧     登录     
document.write(\'<a href="http://www.baidu.com/bdorz/login.gif?login&tpl=
mn&u=\'+ encodeURIComponent(window.location.href+ (window.location.searc
h === "" ? "?" : "&")+ "bdorz_come=1")+ \'" name="tj_login" class="lb">
登录</a>\');\r\n   更多产品               关于百度   About Baidu     
©2017\xa0Baidu\xa0 使用百度前必读 \xa0  意见反馈 
\xa0京ICP证030173号\xa0              \n
\end{minted}

可以发现,里面夹杂了不少非文本内容.因此,我使用了一个更加优秀的Python库:html2text,用法如下:
\begin{minted}{python}
from html2text import HTML2Text
h = HTML2Text()
h.ignore_links = True
h.ignore_images = True
h.handle(raw_html)
\end{minted}

同样以百度首页为例,经过处理后,效果明显优于BeautifulSoup:
\begin{minted}{text}
\n\n新闻 hao123 地图 视频 贴吧 登录 更多产品\n\n关于百度 About Baidu
\n\n(C)2017 Baidu 使用百度前必 读  意见反馈 京ICP证030173号\n\n
\end{minted}
        \subsubsection{中文分词}
由于PyLucene本身对于中文无法正确分词,我们需要借助其他工具来对中文分词,然后再交由PyLucene建立索引.我使用Jieba,一个方便易用
的Python库,来进行分词. 考虑到建立索引的需要,在分词时需要进行一些额外的处理: 去除停用词,长词切分.代码如下:
\begin{minted}{python}
class IndexFiles(object):

    def __init__(self, root, storeDir, analyzer, type="html"):
        # ...
        self.load_stop_words(["CNstopwords.txt", "ENstopwords.txt", ])
        # ...

    def index_html(self, root, writer):

        # ...
        content = jieba.cut_for_search(content)
        content = " ".join(word for word in content
                if word.strip() and word not in self.stop_words)
        # ...

    def load_stop_words(self, file_list):
        self.stop_words = set()
        for file in file_list:
            print("Loading stop words from", file)
            try:
                count = 0
                with open(file, mode="r", encoding="utf8") as f:
                    for line in f:
                        word = line.strip()
                        if word not in self.stop_words:
                            self.stop_words.add(word)
                            count += 1
                        print("\r", count, sep="", end="")
                print("\r{0} word(s) added.".format(count))
            except Exception as e:
                print("Error!")
                print(e)
\end{minted}
        \subsubsection{网页索引}
对于一个网页,根据实验四和五的要求,需要对其内容,标题,网址,存储文件名,存储路径以及主机进行索引.关于content,在上一小节已经叙述了
如何处理;网址,存储文件名,存储路径都可直接从index.txt中读取;而标题,可以使用正则表达式从HTML文档中匹配,
也可以用BeautifulSoup解析网页寻找title标签即可;而主机名,简单的做法是使用urllib.parse里的urlsplit,将url分划后取netloc值.如下:
\begin{minted}{python}
url, path = line.strip().split("\t")
name = os.path.split(path)[-1]
with open(path, mode="r", encoding="utf8") as file:
    content = file.read()
soup = BeautifulSoup(content, "html.parser")
title = soup.title.text if soup.title else "No Title!"
site = urlsplit(url).netloc
\end{minted}

明确了如何对一个网页进行索引后,只需对index.txt中每一行进行处理,读取url和文件路径,
并加入到索引中,这一部分参照PyLucene提供的实例代码\footnote{http://svn.apache.org/viewvc/lucene/pylucene/trunk/samples/IndexFiles.py}即可,在此不表.
    \subsubsection{图片索引}
对图片进行索引的方法与网页索引大致相同,在具体步骤上甚至更为简单. 

对于糗事百科的图片,需要索引的信息有:图片url,图片对应的文字描述,图片所在网页url等.由于在创建索引时对文字描述进行了分词处理,
为了在搜索时能够显示出正确的文字描述,需要额外保存一份原始的文字描述.如下:

\begin{minted}{python}
doc = Document()
doc.add(Field("raw_content", content, t1))
content = " ".join(word for word in jieba.cut_for_search(content)
    if word.strip() and word not in self.stop_words)
doc.add(Field("url", image_url, t1))
doc.add(Field("content", content, t2))
writer.addDocument(doc)
\end{minted}
    \subsection{执行搜索}
        \subsubsection{网页搜索}
在对网页进行搜索时,主要是根据content键中的词频,但根据实验五的要求,还需要支持对site键进行限定.因此,需要
使用\mintinline{python}{BooleanQuery()}实现多条件搜索的功能.

在获取了用户原始输入之后,首先需要从中解析出不同的键值对:
\begin{minted}{python}
def parse_command(command):
    allowed_opts = ["site", ]
    command_dict = dict()
    opt = "content"
    for s in command.split():
        if ":" in s:
            opt, value = s.split(":")[:2]
            opt = opt.lower()
            if opt in allowed_opts and value:
                command_dict[opt] = value
        else:
            command_dict[opt] = command_dict.get(opt, "") + " " + s
    return command_dict
\end{minted}

然后根据返回的命令字典,生成多个查询子句,添加到\mintinline{python}{BooleanQuery.Builder()}中,
最后以此构造出完整的布尔查询.需要注意的是,对于content的搜索,仍需要对其进行分词处理,以保持和创建索引
的一致性.主要代码如下:
\begin{minted}{python}
command_dict = parse_command(command)
querys = BooleanQuery.Builder()
for k, v in command_dict.items():
    if k == "content":
        cutted = [x for x in jieba.cut_for_search(v) if x.strip()]
        v = " ".join(cutted)
    query = QueryParser(k, analyzer).parse(v)
    querys.add(query, BooleanClause.Occur.MUST)
querys = querys.build()
\end{minted}
        \subsubsection{图片搜索}
对图片的搜索可以暂时不考虑site限制(目前仅爬取了单一网站的图片),那么,事情就变得十分简单了.直接贴出图片搜索的
完整代码,如下:
\begin{minted}{python}
def search_image(searcher, analyzer):
    while True:
        print("Hit enter with no input to quit.")
        command = input("Query:")
        os.system("clear")
        if command == "":
            return
        print("Searching for:", command)
        command = " ".join(x for x in jieba.cut_for_search(command))
        query = QueryParser("content", analyzer).parse(command)
        scoreDocs = searcher.search(query, 10).scoreDocs
        print("{} total matching documents.".format(len(scoreDocs)))
        for num, scoreDoc in enumerate(scoreDocs):
            doc = searcher.doc(scoreDoc.doc)
            print("\n#{num}:\nURL:{url}\nContent:{content}\n".format(
                num=num + 1, url=doc.get("url"), content=doc.get("raw_content")))
\end{minted}
    \newpage
    \section{总结与分析}
坦白地说,本次实验是令人沮丧的——我的大部分时间花在了编译安装PyLucene上.实验伊始,我试着按照指引手动编译安装
PyLucene,但多次尝试均无果,不得不自行查找解决方案,其间查阅了PyLucene的官方教程,也粗看了其编译文件的源代码,最终
方能成功安装.

不过尽管如此,这次实验中我将以往实验所写代码进行了重构和整合,再次感受到了面对对象的高效与某些Python独特语法的妙用.
至于PyLucene,我学到了其基本用法,并实现了一个简易的搜索引擎,这一过程也颇有收获.
\end{document}
