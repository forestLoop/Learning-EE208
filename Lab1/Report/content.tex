\documentclass[a4paper]{article}
    \usepackage[UTF8]{ctex}
    \usepackage{minted}
    \usepackage{xcolor}
    \usepackage[colorlinks,linkcolor=black]{hyperref}
    \usepackage[margin=1in]{geometry}
    \usepackage{caption}
    \usepackage{graphicx, subfig}
    \usepackage{float}
    \usepackage{fontspec}
    \setmainfont{Times New Roman}
    \setmonofont{Consolas}
    \definecolor{bg}{rgb}{0.9,0.9,0.9}
    \usemintedstyle{manni}
    \setminted{
    linenos,
    autogobble,
    breaklines,
    breakautoindent,
    bgcolor=bg,
    numberblanklines=false,
    }

\begin{document}
    \tableofcontents
    \newpage
    \section{实验介绍}
        \subsection{实验内容}
本次实验包含3个练习:

1.给定任意网页内容,返回网页中所有超链接的URL(不包括图片地址),并将结果打印至文件res1.txt中,每一行为一个链接地址.

2.给定任意网页内容,返回网页中所有图片地址,并将结果打印至文件res2.txt中,每一行为一个图片地址.

3.给定糗事百科有图有真相任意一页内容,返回网页中图片和相应文本,以及下一页的网址,并将图片地址与相应文本以下述格式打印至文件res3.txt中,
每一行对应一个图片地址与相应文本,格式为:图片地址$\backslash$t相应文本.
        \subsection{实验环境}
操作系统: Ubuntu 18.04 LTS

Python版本: 3.7
        \subsection{所需技能}
本实验的编程部分主要使用Python语言,利用爬虫技术获取网页内容,进而按照一定规则进行解析,提取出所需信息.其中涉及到Python的Requests库(用以
模拟网络请求),BeautifulSoup库(用以解析网页),以及其他的一些库. 当然,对于Linux系统及其命令操作的基本了解,以及网络相关的知识,也有助于更好更快
地完成本次实验.
    \newpage
    \section{实验过程}
        \subsection{准备实验环境}
首先需要说明,我并非故意不遵照PPT的方案,而是在初步尝试之后,发现其有些过时(从推荐使用Ubuntu 14.04可窥见一斑).故为了避免遇到一些
繁琐而无关实验实质的问题,才另辟蹊径,选择采用现在的方法.

首先, 相较于PPT中建议使用的Virtual Box虚拟机, VMWare无疑是更加成熟的解决方案,并且由于VMware Academic Program,其商业化的特点并不构成阻碍.
至于其具体优势,在本实验中体现在可以"一键"安装好操作系统,更加方便的设置共享文件夹等.故,我选择使用VMWare.

其次, Ubuntu 18.04是Ubuntu最新的LTS版本, 与发行于2014年的14.04版本相较, 我认为更加可靠. 毕竟, 使用新版本可以很大程度上避免潜在的Bugs.
故,我选择使用Ubuntu 18.04.

而在Python版本的选择上,由于我曾有过编写爬虫的经验,经历过Python 2对于中文编码的各种不兼容,加上如今使用Python 3的大势所趋,自然,我选择用
最新的Python 3.7,而非建议的Python 2.7.

在明确了以上选择之后, 至于如何安装VMWare,如何在虚拟机中安装Ubuntu,如何设置虚拟机,如何在Linux环境下配置编程所需环境,都不过是细枝末节,
依照所涉及软件的官方指引操作即可,在此不表.

        \subsection{程序总体设计}
分析本次实验的任务,可以发现,涉及到了爬虫系统的两大主要模块:Crawler和Parser.那么,很自然地,运用以往所学的程序设计思想,我计划将这两个模块
分开而各自单独实现在一个Python文件中.具体而言,例如Parser,既要实现一个基本的解析器用以提取网页中的URL和图片,又需要针对糗事百科实现一个专门
的解析器,那么,面对对象的程序设计方法就有了用武之地: 先实现一个BaseParser,再以此派生出糗事百科的Parser.

在实现这两个模块之后, 再着手于本次实验的具体任务, 只需编写一个主程序(main.py),在其中实例化所需的Crawler和Parser,执行相应的任务即可.    
        \subsection{编程实现}
            \subsubsection{Crawler}
在本次实验中,仅仅涉及到基本的GET请求以及添加请求的Headers信息,故实现一个BaseCrawler即可. 不过,在具体的实现上,我使用了Requests库而非
Urllib库,原因在于Requests库具有更简洁的操作方式,同时支持Session(以便于处理某些需要登录的站点,尽管本次未涉及). 
关键代码(位于crawlers.py文件)如下:
\begin{minted}{python}
import requests
class BaseCrawler(object):

    def __init__(self, headers=None):
        self.new_session()
        self.headers = headers

    def get_html(self, url, headers=None):
        assert self.session != None
        headers = headers or self.headers  # if headers are not specified for this single request, use the global headers
        content = self.session.get(url, headers=headers).content
        decoded_content = content.decode("utf8")  # maybe not utf-8 for some sites?
        return decoded_content

    def new_session(self):
        self.session = requests.Session()
\end{minted}
在我的实现中,若在单次请求时未指定Headers,则沿用对象创建时设定的Headers. 至于网页编码, 尽管UTF-8编码可以应对多数网页,
但在实际测试中我确实发现了一些网页并非如此,所以,在后续的实验中,可能需要对此进行一些修改.
            
            \subsubsection{Parser}
如前所言,主要有BaseParser和QSBKParser.在BaseParser中,利用BeautiSoup以及正则表达式实现了提取网页中的URL和图片地址的功能.但对于
实际的网页,\mintinline{html}{<a>}标签的\mintinline{html}{href}属性和\mintinline{html}{<img>}标签的\mintinline{html}{src}
属性中的内容并不一定总是标准的URL,很有可能是相对路径,甚至根本不是链接(如某些链接实际是调用Javascript脚本). 

故,需要对提取到的可能为链接的内容进行处理. 具体而言, 即去除非法链接,将相对路径补全为标准链接(此处用到了urllib.parser中的urljoin).
而对于图片,还需判断其后缀是否为常见的图片格式.具体实现如下:

\begin{minted}{python}
from bs4 import BeautifulSoup
from urllib.parse import urljoin
import re
class BaseParser(object):

    def __init__(self):
        self.standard_url_pattern = re.compile(r"^(https?:)?//[^\s]*$")  # https:// or http:// or //
        self.relative_path_pattern = re.compile(r"^\.{0,2}/[^\s]*$")  # / or ./ or ../
        self.common_img_formats = ("bmp", "jpg", "jpeg", "png", "gif")

    def parse_url(self, html_content, current_url=None):
        soup = BeautifulSoup(html_content, features="html.parser")  # specify features to avoid potential warnings
        url_set = set()
        for a in soup.findAll("a"):
            maybe_url = a.get("href", "")
            final_url = self.handle_url(maybe_url, current_url)
            if final_url:
                url_set.add(final_url)
        return url_set

    def parse_img(self, html_content, current_url=None):
        soup = BeautifulSoup(html_content, features="html.parser")
        img_set = set()
        for img in soup.findAll("img"):
            maybe_img = img.get("src", "")
            final_img = self.handle_img(maybe_img, current_url)
            if final_img:
                img_set.add(final_img)
        return img_set

    def handle_url(self, maybe_url, current_url=None):
        maybe_url = maybe_url.strip()
        if self.standard_url_pattern.match(maybe_url) or self.relative_path_pattern.match(maybe_url):
            if current_url:
                return urljoin(current_url, maybe_url)
            else:
                return maybe_url  # without the current url, I can't turn a relative path into a standard url
        else:
            return None

    def handle_img(self, maybe_img, current_url=None):
        final_url = self.handle_url(maybe_img, current_url)
        if not final_url:  # make sure that it's a valid url
            return None
        last = final_url.split(".")[-1]
        if last in self.common_img_formats:
            return final_url
        else:
            return None
\end{minted}

需要注意的是, 将相对路径转换为绝对路径需要已知当前路径,即代码中的\mintinline{python}{current_url}.在调用时需要指定值,否则无法处理相对路径,
只能将其不做处理原样返回.

以及另一个细节, 在使用BeautifulSoup时, 若不指定\mintinline{python}{features="html.parser"},则会产生Warnings,虽不影响程序运行,
但产生了大量无关输出.

在BaseParser的基础上,以其为基类,派生出QSBKParser.因此可以复用基类的方法,只需为其单独实现针对糗事百科页面的解析功能.代码如下:
\begin{minted}{python}
class QSBKParser(BaseParser):

    def __init__(self):
        super(QSBKParser, self).__init__()

    def parse_page(self, html_content, current_url=None):
        soup = BeautifulSoup(html_content, features="html.parser")
        docs = dict()
        for single_post in soup.findAll("div", {"id": re.compile(r"qiushi_tag_\d+")}):
            qiushi_tag = single_post["id"].split("_")[-1]
            content = single_post.find("div", {"class": "content"}).span.text.strip()
            maybe_img = single_post.find("div", {"class": "thumb"}).a.img["src"]
            img_url = self.handle_img(maybe_img, current_url)
            docs[qiushi_tag] = {
                "content": content,
                "img_url": img_url,
            }
        maybe_next_page = soup.find("span", {"class": "next"}).parent["href"]
        next_page = self.handle_url(maybe_next_page, current_url)
        return docs, next_page
\end{minted}
            \subsubsection{Main}
由于本次实验要求解析命令行参数,如通过命令行设定要解析的网址,我使用了sys库以及getopt库.前者用以获取运行程序的参数,后者用以解析.

考虑到3个练习具有共性,为减少冗余代码,我并未为每个练习单独编写程序,而是在调用时传入\mintinline{bash}{--task(-t)}参数来选择. 除此之外,
还支持\mintinline{bash}{--url(-u)}来设定网址,\mintinline{bash}{--output(-o)}来设定输出文件,以及\mintinline{bash}{--help(-h)}
来输出帮助信息.

则main.py的主体代码(略去具体函数实现)如下:

\begin{minted}{python}
def main(argv):
    url = "https://www.baidu.com"
    task = 0
    output_file = None
    try:
        opts, _ = getopt.getopt(argv, "u:t:o:h", ["url=", "task=", "output=", "help"])
    except getopt.GetoptError:
        print("Invalid Arguments!")
        opts = (("-h", None),)
    for opt, arg in opts:
        if opt in ("-h", "--help"):
            show_help()
            exit()
        elif opt in ("-u", "--url"):
            url = arg
        elif opt in ("-t", "--task"):
            task = int(arg)
        elif opt in ("-o", "--output"):
            output_file = arg
    if task == 1:
        get_urls(url, output_file or "res1.txt")
    elif task == 2:
        get_imgs(url, output_file or "res2.txt")
    elif task == 3:
        get_jokes(url, output_file or "res3.txt")

if __name__ == '__main__':
    main(sys.argv[1:])
\end{minted}

在运行时,可采取如下方式:
\begin{minted}{bash}
    python main.py -u www.baidu.com -o baidu.txt -t 1
\end{minted}
            \subsubsection{Unit Test}
由于采用了模块化编程,为了减少Bug,抑或更方便地发现Bug,对单个模块进行测试是必要的. 故我在crawlers.py和parsers.py中,均编写了用以简单测试的
代码. 例如,在crawlers.py中:
\begin{minted}{python}
def unit_test():
    print("Unit Test Begins.")
    test_crawler = BaseCrawler()
    print("Try to get the content of https://keithnull.top/")
    html = test_crawler.get_html("https://keithnull.top/")
    assert html.find("Keith") != -1
    print("Success")

if __name__ == '__main__':
    unit_test()
\end{minted}
    \newpage
    \section{总结与分析}
在这次实验之前,我其实已有过编写爬虫的经验(高中时期,用Python 2编写了一个爬取校内信息并存储在数据库中的爬虫),正是如此,我意识到编写一个爬虫不仅仅是
模拟请求然后解析数据,而是需要对代码有合理的组织,否则只会把自己困缚,难以调试和维护. 所以,本次实验我并没有完全依照PPT的建议,
而是自行改变了很多.或许这些"自作主张"正朝着更合理的方向,又或许毫无意义,但都是我思考之后的决定.

这次实验给了我一个从零开始编写爬虫的机会,有些时候我愿意按着自己的想法进行,无谓对错,我相信,思考与实践的过程即是收获.
\end{document}
